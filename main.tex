\documentclass[12pt,twoside,openright,a4paper]{book}
\usepackage{graphicx}
\usepackage{amsmath}
\usepackage{amsthm}
\usepackage{makeidx}
\makeindex

\newtheorem{theorem}{Theorem}[chapter]
\newtheorem{lemma}[theorem]{Lemma}
\newtheorem{corollary}[theorem]{Corollary}
\newtheorem{conjecture}{Conjecture}[chapter]
\newcommand{\initch}{\text{ch}_0}
\newcommand{\finch}{\text{ch}}
\newcommand{\GG}{\mathcal{G}}

\begin{document}
\pagestyle{empty}
\begin{titlepage}

\begin{center}

\vspace{10cm}

{\large\bf A Coloring Book}

\end{center}
\end{titlepage}

\newpage

\pagestyle{plain}
\pagenumbering{roman}
\tableofcontents

\newpage

\pagestyle{headings}
\pagenumbering{arabic}
\setcounter{page}{1}

\chapter*{Introduction}
\addcontentsline{toc}{chapter}{Introduction}

\section{Scope of the book}

(sparse graphs, vertex coloring), target audience (PhD level,
familiarity with graph theory and basic coloring results expected, otherwise self-contained).

\section{Basic definitions and notation}

% claims: A, B, C, ... local
% labels: "lemma:zzz", "thm:zzz"
% numbered definitions only for important concepts

Notation for number of vertices/edges/faces of a graph $G$:
$|V(G)|$, $|E(G)|$, $|F(G)|$.  Use $n$ and $m$ for vertices and edges when repeated.
Graphs have no loops or parallel edges by default, multigraphs may have parallel edges but still no loops.

Colorings: $\varphi$, $\psi$, \ldots

$[k]=\{1,\ldots,k\}$.

$P_k$ denotes a path on $k$ vertices, its length is $k-1$.

$C_k$ denotes a cycle on $k$ vertices, its length is $k$.  Length of a cycle $C$ is denoted by $|C|$.

Girth (no special notation)?

Graphs on surfaces: genus (Euler by default), representativity and edge-width, contractible and
noncontractible curves (separating vs non-separating, one-sided vs two-sided).

Planar vs plane graphs.

Face of an embedded graph $G$ is a maximal connected subset of the surface minus $G$.
A face is \emph{$2$-cell} if it is homeomorphic to an open disk, \emph{closed $2$-cell}
if additionally bounded by a cycle.  Embedding is $2$-cell if all faces are $2$-cell.
\emph{Facial walk} of $f$ is a closed walk in $G$ tracing
a component of the boundary of $f$, the length $|f|$ of $f$ is the sum of lengths of its facial
walks.

Surfaces: $\Sigma$, $\Pi$; Euler genus $g(\Sigma)$.

Other notation (to be moved to where it becomes relevant).

critical for $k$-coloring/$k$-list-coloring

list chromatic number, $k$-list-colorable

If $G$ is $H$-critical,then $G\neq H$.

$\initch$ for initial charge, $\finch$ for final charge.  Discharging to non-negative.

(maximum) average degree/degeneracy/coloring number

\section{Overview of Results}

Note: How much overview here?

\part{Standard Methods}

Overview of Standard Methods (with examples and comments on common tricks and pitfalls):

Grotzch's theorem as a non-trivial motivating example

\chapter{Reducible Configurations and Discharging}

% NEEDED: list coloring, greedy algorithm, Brooks for list coloring (Gallai trees)

% ZD
Suppose we want to show that graphs from some hereditary class $\GG$ are $k$-colorable.  Clearly, we can
restrict our attention to graphs from $\GG$ of minimum degree at least $k$: if a graph $G\in \GG$
contains a vertex $v$ of degree less than $k$, it suffices to find a $k$-coloring of the graph $G-v$
which also belongs to $\GG$, and then give $v$ a color different from the colors of its neighbors,
which is always available since $v$ has less than $k$ neighbors.

More formally, we are constructing a proof by contradiction.  Suppose that $G$ is a graph belonging to $\GG$
with the smallest number of vertices that is not $k$-coloriable.  Hence, $G$ is a hypothetical smallest counterexample
to the claim that all graphs from $\GG$ are $k$-colorable.
By the previous paragraph, we see that $G$ does not contain a vertex of degree at most $k-1$.
Substructures with the property that they cannot appear in any smallest counterexample
are called \emph{reducible configurations}, and they form a basis of many arguments in graph coloring.

In the most direct application, we can be able to identify a set of reducible configurations such that
at least one of them must necessarily appear in any graph from $\GG$, thus excluding the existence of
a (smallest) counterexample and showing that indeed, all graphs from $\GG$ are $k$-colorable.
In more advanced applications, reducible configurations can be used to restrict the properties
of the smallest counterexample, and a contradiction is obtained by different means afterwards, typically
by showing through another method that all graphs satisfying these restrictions are $k$-colorable.

There are many tricks used in showing that a configuration is reducible, which we will discuss some
of them throughout this chapter and elsewhere in the book. However, the basic idea almost always resembles
our starting example: the configuration is removed (and possibly some further local changes are
performed in the graph), and then a coloring of the modified graph is transformed in a coloring of the
original graph.  Reducible configurations usually involve vertices whose degree is
smaller than or close to the number of colors $k$ we are using, since a coloring of the rest of the graph
can be extended to such vertices (nearly) greedily.

Consequently, reducible configurations are easiest to find in graph classes $\GG$ that are sparse, in the
sense that the graphs in $\GG$ have average degree close to $k$. To prove that each graph in such a class $\GG$
contains one of the reducible configurations, one needs to argue that their absence would force the graph to be
too dense.  The \emph{discharging method} is a systematic way to organize such an argument.  In the most straightforward
form, a real number called \emph{charge} is assigned to vertices (and in the case of embedded graphs, possibly faces)
of the graph, so that the sum of the charges is negative; for example, if graphs in $\GG$ have average degree less than $d$,
a possible assignment of charge to a vertex $v$ is $\deg v-d$.  The charge is then redistributed according to some local
rules, without changing its total amount.  As a result, there exists a vertex (or face) whose final charge is negative,
and based on the discharging rules, we argue that this is only possible if a reducible configuration appears in its neighborhood.
Of course, this description omits the most important aspect: how to choose the discharging rules?
While this seems a daunting task at first, with a bit of experience this becomes a fairly mechanical process of trial and error
(which can in fact be automatized to some extent), as we will discuss later in this chapter.

Although reducible configurations and discharging method are independent tools that have many applications on their
own, in the coloring context they are most commonly applied together, and so we devote this chapter to both of them.

\section{Overview of the Method}

Let us now flesh out the ideas from the introduction a bit.  While discharging and reducible configurations
are applicable in many contexts, they are at their best in the realm of graphs drawn on a fixed surface, and
especially for planar graphs.  This is primarily due to the well-known relationship between the numbers
of their vertices, edges, and faces.

\begin{theorem}[Euler's formula]\label{thm:eulerfla}
If $G$ is a graph embedded in a surface of genus $g$, then
$$|E(G)|\le |V(G)|+|F(G)|+g-2,$$
with equality if and only if the embedding of $G$ is $2$-cell.
\end{theorem}

Let us start with a mandatory toy example: 5-colorability of planar graphs.

\begin{theorem}\label{thm:planar5col}
Every planar graph is $5$-colorable.
\end{theorem}
\begin{proof}
Suppose for a contradiction that this is not the case, and let $G$ be a non-$5$-colorable plane graph
with the smallest number of vertices.

Clearly, $G$ is connected and it has minimum degree at least $5$.  We can strengthen this claim: $G$ does not
contain any vertex of degree $5$.  Indeed, suppose that $v\in V(G)$ has degree $5$.  Since $G$ is planar, it does
not contain $K_6$ as a subgraph, and thus $v$ has two non-adjacent neighbors $x$ and $y$.  Let $G'$ be the graph obtained
from $G$ by removing $v$ and identifying vertices $x$ and $y$ to a single vertex $z$.  Note that $G'$ is a plane graph,
and by the minimality of $G$, we conclude that $G'$ has a $5$-coloring $\varphi$.  We can extend this $5$-coloring to $G$,
by giving $x$ and $y$ the color $\varphi(z)$, and by coloring $v$ greedily (there are $5$ available colors, but only
$4$ of them are used at the neihbors of $v$, since $x$ and $y$ have the same color).  This contradicts the assumption
that $G$ is not $5$-colorable, and thus we conclude that $G$ cannot contain a vertex of degree $5$.

Let us now give each vertex $v\in V(G)$ charge $\initch(v)=\deg(v)-6$ and each face $f\in F(G)$ charge $\initch(f)=2|f|-6$.
Since all vertices of $G$ have degree at least $6$, their charge is non-negative.  Furthermore, $G$ clearly has at least $6$
vertices, and since it is connected, all its faces have length at least $3$ and their charge is non-negative.
However,
$$\sum_{v\in V(G)} \initch(v)=\left(\sum_{v\in V(G)} \deg(v)\right)-6|V(G)|=2|E(G)|-6|V(G)|$$
and
$$\sum_{f\in F(G)} \initch(f)=\left(2\sum_{f\in F(G)} |f|\right)-6|F(G)|=4|E(G)|-6|F(G)|,$$
and thus the sum of charges of vertices and faces is
$$6(|E(G)|-|V(G)|-|F(G)|)=-12$$
by Euler's formula.  This is a contradiction, showing that no non-$5$-colorable plane graph exists.
\end{proof}

For the demonstration purposes, the preceding example has a disadvantage of
having only a trivial discharging phase, with no discharging rules.  Let us
turn this to our advantage and discuss two aspects seen in the proof: its
overall structure and the choice of the assignment of charges.

We see that the proof is cleanly split to two parts: first we identify the reducible configurations, then we show
that they need to appear in each planar graph.  This makes for a clear presentation, but it hides the process of
how we devised the two parts, which usually is not in isolation; more on that in Section~\ref{sec:rules}.

Both parts are presented as proofs by contradiction: we consider a smallest counterexample in the first part,
and we argue that since we have no reducible configuration the sum of charges would have to be non-negative.
This is not the only possiblity, and both parts can be presented in a more ``positive'' manner.

\begin{proof}[Proof of Theorem~\ref{thm:planar5col}, alternative formulation]
We prove the claim by induction on the number of vertices.  So, consider some plane graph $G$, and assume
that every planar graph with fewer than $|V(G)|$ vertices is $5$-colorable.

If $G$ is disconnected, we can $5$-color its components separately by the induction hypothesis, and thus $G$
is $5$-colorable.  Hence, we can assume that $G$ is connected.  Also, if $|V(G)|\le 5$, then we can give all
vertices distinct colors.  Hence, assume that $|V(G)|\ge 6$.

Let us assign charges to vertices and faces in the same way as in the first proof of the theorem.
Since their sum is negative and all faces have non-negative charge, there exists a vertex $v\in V(G)$
such that $\initch(v)<0$, and thus $\deg(v)\le 5$.

If $v$ has degree at most $4$, then a $5$-coloring of $G-v$ which exists by the induction hypothesis can be extended to $G$ by giving $v$ a color
different from the colors of its neighbors.  And, if $v$ has degree $5$, then we can $5$-color by
the induction hypothesis the graph $G'$ constructed in the same way as in the first proof of the theorem, and extend its
$5$-coloring to $G$.
\end{proof}

Of course, both proofs are equivalent, and it is a matter of taste which style of presentation one chooses.
In our experience, the latter style is more convenient when one wants to discuss the algorithmic aspects of the proof.
The former style has the advantage that we can speak about the properties of one fixed object (a hypothetical minimal
counterexample $G$), and say things like ``$G$ has minimum degree at least $k$ and does not contain this-and-that'';
the respective formulations in the latter style tend to be more cumbersome.

\subsection{Initial assignment of charge}\label{ssec:initch}

How do we decide what initial charge we assign to vertices and faces?  Since we want to take advantage of Euler's formula,
there is actually not much freedom, and when the discharging method is used for embedded graphs, the assignment
of charges almost exclusively follows one of the options from the following lemma.

\begin{lemma}\label{lemma:initch}
Let $a\ge 0$ and $b>0$ be real numbers, and let $G$ be a graph with $2$-cell embedding in a surface of genus $g$.
Let $\initch(v)=a\deg(v)-2b$ for $v\in V(G)$ and $\initch(f)=(b-a)|f|-2b$ for $f\in F(G)$.
Then
$$\sum_{v\in V(G)}\initch(v)+\sum_{f\in F(G)}\initch(f)=2b(g-2).$$
\end{lemma}
\begin{proof}
By Euler's formula, we have
\begin{align*}
\sum_{v\in V(G)}\initch(v)+\sum_{f\in F(G)}\initch(f)&=\left(-2b|V(G)|+a\sum_{v\in V(G)}\deg(v)\right)+\left(-2b|F(G)|+(b-a)\sum_{f\in F(G)}|f|\right)\\
&(-2b|V(G)|+2a|E(G)|)+(-2b|F(G)|+2(b-a)|E(G)|)\\
&=2b(|E(G)|-|V(G)|-|F(G)|)=2b(g-2).
\end{align*}
\end{proof}

The most common choices of charge for vertices $v$ and faces $f$ are the following:
\begin{itemize}
\item $\initch(v)=\deg(v)-6$ and $\initch(f)=2|f|-6$, obtained by setting $a=1$ and $b=3$: This is the charge we used in the proof of Theorem~\ref{thm:planar5col},
and for connected plane graphs, it sums to $-12$.  The charge of all faces is nonnegative, putting the focus on vertices.  This tends to be a good
choice when proving results for unrestricted plane graphs.
\item $\initch(v)=\deg(v)-4$ and $\initch(f)=|f|-4$, obtained by setting $a=1$ and $b=2$: For connected plane graphs, the charge sums to $-8$.
Only $3$-faces have negative charge, which makes this a good choice for triangle-free graphs or graphs with restrictions on triangles.
\item $\initch(v)=2\deg(v)-6$ and $\initch(f)=|f|-6$, obtained by setting $a=2$ and $b=3$: For connected plane graphs, the charge sums to $-12$.
and this choice is convenient when dealing with $3$-regular graphs (so that vertices have no charge), or sometimes when dealing with graphs of large girth.
\end{itemize}
Of course, other choices may be useful in other special situations. The proper
choice of charge can greatly simplify the discharging rules.  However, let us
remark that at least in principle, all the charge choices are equivalent.
Indeed, suppose that we have assigned charge $\initch(v)=a\deg(v)-2b$ for $v\in V(G)$ and $\initch(f)=(b-a)|f|-2b$ for $f\in F(G)$.
Consider any other real numbers $a'\ge 0$ and $b'>0$.
Now, let each vertex send $\gamma=\frac{ab'-a'b}{b'}$ units of charge to each incident face, and let $\finch$ denote the resulting assigment of charges.
We have $\finch(v)=(a-\gamma)\deg(v)-2b=\frac{b}{b'}(a'\deg(v)-2b')$ and $\finch(f)=(b-a+\gamma)|f|-2b=\frac{b}{b'}((b'-a')|f|-2b')$,
which up to scaling is the same charge we would obtain if we instead of $a$ and $b$ chose $a'$ and $b'$.
Hence, if a discharging argument is possible with the latter choice, it is also possible with the former one, by adding the discharging rule of
moving $\gamma$ units of charge from each vertex to each incident face.
In other words, choosing a ``wrong'' charge may make the discharging rules more complicated, but not impossible to find.

Let us remark that some authors prefer to negate the charge assignments we suggest, so that the sum of charges is positive and
after redistribution using the discharging rules, a reducible configuration
is then found near the vertices or faces of positive charge.  This convention can be mostly found
in older works, most notably in the proofs of Four Color Theorem~\cite{AppHak1,rsst}.  As this is purely a matter of taste
and nowadays the choice of charge summing to negative is more popular, this will be the convention used in this book.

\section{A more substantial example}

We now give a further example demonstrating more of the aspects of the method.  Specifically, we will show that all
projective-planar graphs without $4$-cycles are $4$-colorable.  Let us start with the reducible configurations.

\begin{lemma}\label{lemma:no4cproj-redu}
Let $G$ be a graph with no $4$-cycles drawn in the projective plane, such that every projective-planar graph
with no $4$-cycles and with fewer than $|V(G)|$ vertices is $4$-colorable.  If $G$ is not $4$-colorable, then
\begin{itemize}
\item[\textrm{(a)}] $G$ is connected;
\item[\textrm{(b)}] the minimum degree of $G$ is at least $4$;
\item[\textrm{(c)}] if $f$ is a $5$-face of $G$ sharing an edge $v_1v_2$ with a $3$-face $g$ and all vertices
incident with $f$ and $g$ except for $v_1$ have degree $4$, then $v_1$ has degree at least $6$; and,
\item[\textrm{(d)}] if $v\in V(G)$ has degree $6$ and is incident with a $3$-face $g_1$, a $5$-face $f_1$,
a $3$-face $g_2$, and a $5$-face $f_2$ in order, then at least one vertex incident with these faces other than $v$ has
degree at least $5$.
\end{itemize}
\end{lemma}
\begin{proof}
For (a), if $G$ were a disjoint union of non-empty graphs $G_1$ and $G_2$, then both $G_1$ and $G_2$ would be
$4$-colorable by the assumption, and thus $G$ would also be $4$-colorable.  Similarly, for (b), if $G$ contained a vertex $v$ of
degree at most $3$, then $G-v$ is $4$-colorable by the assumption and the coloring can be extended to $G$ by giving $v$ a color
different from the colors of its neighbors.

For (c), let $K=v_1v_2\ldots v_5$ be the boundary walk of $f$.  Let us first consider the case that the face $f$ is not bounded by a $5$-cycle.
Since $G$ has no loops or parallel
edges, an edge $e$ appears twice in the boundary of $f$, and the remaining three edges form a triangle.  Note that $e\neq v_1v_2$,
since the edge $v_1v_2$ is incident with a face $g$ distinct from $f$.  Also, the only triangle containing the edge $v_1v_2$
is the one bounding $g$, since $G$ does not contain a $4$-cycle.  It follows that either the path $v_1v_2v_3$ or the path $v_5v_1v_2$
is contained in the boundaries of both $f$ and $g$, and thus either $v_1$ or $v_2$ has degree two, which is a contradiction.

Hence, $K$ is a cycle, and since $G$ has no $4$-cycles, $K$ is induced.  Let $T=v_1v_2v_6$ be the triangle bounding the face $g$.
Since $K$ is an induced cycle, we have $v_6\not\in V(K)$, and since $G$ contains no $4$-cycles, $v_6$ has no neighbor in $K$ other
than $v_1$ and $v_2$.  Suppose for a contradiction that $\deg(v_1)\le 5$.  By the assumptions, there exists a $4$-coloring $\varphi$
of $G-\{v_1,\ldots, v_6\}$.  For $i\in[6]$, let $L(v_i)$ denote the subset of $[4]$ consisting of the colors not
appearing on the neighbors of $v_i$ in the coloring $\varphi$; we have $|L(v_i)|\ge 2$ for $i\in\{1,3,4,5,6\}$ and $|L(v_2)|=3$.
To extend the coloring $\varphi$ to a $4$-coloring of $G$, it suffices to color $K\cup T$ from the lists given by $L$.

First, let us choose a color $c_1\in L(v_2)$ such that $|L(v_3)\setminus\{c\}|\ge 2$.  If we can color $v_1$ and $v_6$ by colors
different from $c_1$, we can then color $v_2$ by $c_1$ and greedily color $v_5$, $v_4$, and $v_3$ in order by colors from their lists different from
the colors of their neighbors.  Otherwise, we have $L(v_1)=L(v_6)=\{c_1,c_2\}$ for some color $c_2$.
Let $c_3\in L(v_2)$ be a color different from $c_1$ and $c_2$.  We color $v_2$ by $c_3$ and greedily color $v_3$, $v_4$, $v_5$, $v_1$, and $v_6$
in order.  It follows that $G$ is $4$-colorable, which is a contradiction; hence, (c) holds.

Finally, let us consider (d).  Let $vv_1v_2$, $vv_2v_3v_4v_5$, $vv_5v_6$, and $vv_6v_7v_8v_9$ be the facial walks of
$g_1$, $f_1$, $g_2$, and $f_2$.  Observe that since $G$ has no parallel edges, no $4$-cycles, and minimum degree at least $4$,
the vertices $v_1$, \ldots, $v_9$ are pairwise distinct.  By the assumptions, there exists a $4$-coloring $\varphi$ of $G-\{v,v_1,\ldots,v_9\}$.
Note that $v$ has only one neighbor $y$ colored by $\varphi$, and $v_1$ has at most two neighbors $y_1$ and $y_2$ that are
colored by $\varphi$.  Hence, we can choose a color $c\in[4]\setminus\{\varphi(y)\}$ such that $|[4]\setminus\{c,\varphi(y_1),\varphi(y_2)\}|\ge 2$.
Color $v$ by the color $c$, then color vertices $v_9$, $v_8$, \ldots, $v_1$ greedily in order.
\end{proof}

Now, we can proceed with the discharging phase.

\begin{theorem}\label{thm:no4cproj}
Every projective-planar graph without $4$-cycles is $4$-colorable.
\end{theorem}
\begin{proof}
Suppose for a contradiction this is not the case, and let $G$ be a
non-$4$-colorable graph drawn in the projective plane with
the smallest number of
vertices.  By Lemma~\ref{lemma:no4cproj-redu}(a), $G$ is connected.
Let us assign initial charge $\initch(v)=\deg(v)-4$ to each vertex $v$ of $G$,
and initial charge $\initch(f)=|f|-4$ to each face $f$ of $G$.

If the embedding of $G$ is $2$-cell, then the sum of the initial charges
is $-4$ by Lemma~\ref{lemma:initch}.  If the embedding is not $2$-cell, then
since $G$ is connected, there exists a unique face of $G$ whose interior contains
the crosscap of the projective plane, and $G$ is planar and the sum of the initial charges is $-8$.
In either case, the total amount of charge is negative.

Next, we redistribute the charge according to the following rules.
For a $3$-face $g$, let $r(g)$ denote the number of incident vertices of degree
at least $5$.
\begin{itemize}
\item[(R1)] A vertex $v$ of degree at least $5$ incident with a $3$-face $g$ sends $1$ to $g$
if $\deg(v)\ge 7$, or if $\deg(v)=6$ and $v$ is incident with at most two triangles,
or if $\deg(v)=6$ and $g$ is the only triagle incident with $v$ satisfying $r(g)=1$.
The vertex $v$ sends $2/3$ to $g$ if $\deg(v)=6$ and $v$ is incident with three $3$-faces $g$, $g_1$, and $g_2$
such that $r(g)=r(g_1)=1$.  Otherwise, $v$ sends $1/2$ to $g$.
\item[(R2)] If a face $f$ shares an edge $uv$ with a $3$-face $g$ such that $\deg(u)=\deg(v)=4$,
then $f$ sends $1/3$ to $g$ if $r(g)=0$, and $f$ sends $1/5$ to $g$ if $r(g)=1$.
\item[(R3)] If a face $f$ shares an edge $uv$ with a $3$-face $g$ such that $r(g)=1$ and $\deg(u)=4$ and $\deg(v)\ge 5$,
then $f$ sends $3/20$ to $g$ if $|f|\ge 6$, or $|f|=5$ and $\deg(v)=5$, or $|f|=5$ and $\deg(v)=6$ and $f$ is incident with a vertex of degree
at least $5$ distinct from $v$.
\end{itemize}
Let $\finch$ denote the final charge after performing the redistribution.
Note that the total amount of charge is unchanged, and thus the sum of
final charges is negative.

We argue that each vertex and face has a non-negative final charge,
which gives a contradiction.  Let us start with vertices: By Lemma~\ref{lemma:no4cproj-redu}(b),
each vertex $v$ of $G$ has degree at least $4$.  If $\deg(v)=4$, then $v$ sends no charge
and $\finch(v)=\initch(v)=0$.  Suppose now that $\deg(v)\ge 5$, and let $t$ denote the number
of $3$-faces incident with $v$.  Since $G$ contains no $4$-cycles, no two $3$-faces are consecutive
in the cyclic order around $v$, and thus $t\le \lfloor\deg(v)/2\rfloor$.
If $\deg(v)=5$, then $t\le 2$ and $v$ sends $1/2$ to each incident $3$-face by (R1), and thus
$\finch(v)=\initch(v)-t/2=1-t/2\ge 0$.  If $\deg(v)\ge 7$, then $v$ sends $1$ to each incident $3$-face by (R1),
and thus
$\finch(v)=\initch(v)-t\ge \deg(v)-4-\lfloor\deg(v)/2\rfloor=\lceil\deg(v)/2\rceil-4\ge 0$.
So, suppose that $\deg(v)=6$.  If $t\le2$, then $\finch(v)=\initch(v)-t\ge 0$.  If $t=3$, then let $g_1$, $g_2$, and $g_3$
be the $3$-faces incident with $v$ such that $r(g_1)\le r(g_2)\le r(g_3)$.  If $r(g_2)\ge 2$, then $v$ sends $1/2$ to $g_2$ and to $g_3$,
and at most $1$ to $g_1$, hence $\finch(v)\ge\initch(v)-2=0$.  If $r(g_2)=1$, then $v$ sends at most $2/3$ to each incident $3$-face,
and $\finch(v)\ge\initch(v)-2=0$.

Next, consider the charge of the faces.  Let $g$ be a $3$-face; since $G$ does not contain $4$-cycles
and has minimum degree greater than $2$, all faces that share an edge with $g$ have length at least $5$,
and thus $g$ sends no charge.  If $r(g)\ge 2$, then $g$ receives at least $1/2$ from each incident vertex of degree at least $5$ by (R1)
and thus $g$ receives at least $1$ in total.
If $r(g)=0$, then $g$ receives $1/3$ from each face with that it shares an edge by (R2).
Hence, suppose that $r(g)=1$, and let $v$ be the vertex of degree at least $5$ incident with $g$.  Note that
$g$ receives $1/5$ from the face opposite to $v$ by (R2).  If $\deg(v)=5$, then $g$ receives $1/2$ from $v$ by (R1)
and $3/20$ from each of the two faces with that $g$ shares an edge incident with $v$ by (R3), and the total amount received
is $1/5+1/2+2\times 3/20=1$.  If $\deg(v)=7$, or $\deg(v)=6$ and $v$ is inicident with at most two $3$-faces,
or $\deg(v)=6$ and $g$ is the only $3$-face with $r(g)=1$ incident with $v$, then $g$ receives $1$ from $v$ by (R1).
Finally, suppose that $\deg(v)=6$, $v$ is incident with three $3$-faces, and a $3$-face $g_1\neq g$ incident with $v$ satisfies $r(g_1)=1$.
In this case, $v$ sends $2/3$ to $g$.
Let $f_1$, $g$, $f_2$, $g_1$ be faces incident with $v$ in order according to the drawing of $G$.
By Lemma~\ref{lemma:no4cproj-redu}(d), it cannot be the case that both $f_1$ and $f_2$ are $5$-faces whose only incident vertex
of degree at least $5$ is $v$, and thus either $f_1$ or $f_2$ sends $3/20$ to $g$ by (R3).  The total amount sent to $g$
is at least $1/5+2/3+3/20>1$.
In all the cases, the final charge of $g$ is at least $\initch(g)+1=0$.

Finally, let $f$ be a face of length at least $5$.  If $|f|\ge 6$, then $\finch(f)\ge \initch(f)-|f|/3=\frac{2}{3}|f|-4\ge 0$
by (R2) and (R3).  Hence, suppose that $|f|=5$.  If $f$ sends at most $1/5$ to each incident $3$-face by (R2), then
$\finch(f)\ge \initch(f)-1=0$.  Hence, suppose that $f$ sends $1/3$ to an incident $3$-face $g$ with $r(g)=0$ by (R2).
By Lemma~\ref{lemma:no4cproj-redu}(c), not all vertices incident with $f$ have degree $4$.  Let $v_1v_2v_3v_4v_5$ be the boundary
walk of $f$, where $\deg(v_5)\ge 5$.  If at least one of $v_1$, \ldots, $v_4$ has degree at least $5$, then
$f$ sends at most $\max(2\times 1/3+2\times 3/20, 1/3+4\times 3/20)<1$ by (R2) and (R3), and $\finch(f)>\initch(f)-1=0$.
Hence, suppose that $\deg(v_1)=\ldots=\deg(v_4)=4$.  Let $g_1$ and $g_2$ be the faces sharing edges $v_1v_5$ and $v_4v_5$ with $f$,
respectively.  If $f$ sends charge to neither $g_1$ nor $g_2$ by (R3), then $f$ sends at most $3\times 1/3$ by (R2) and (R3),
and $\finch(f)\ge \initch(f)-1=0$.  By symmetry, we can assume that $f$ sends positive amount of charge to $g_1$, and thus $g_1$
is a $3$-face with $r(g_1)=1$.  Since the only vertex incident with $f$ of degree at least $5$ is $v_5$, the rule (R3)
applies only if $\deg(v_5)=5$.  However, this is not possible by Lemma~\ref{lemma:no4cproj-redu}(c).
\end{proof}

In the next section, we explain how we came up with the reducible configurations and arcane-looking rules (R1)--(R3).
However, before that, let us comment on errors that often appear (even in some published) proofs using the method of reducible
configurations.

\subsection{Common errors}
Remark on faces bounded by non-cycles.

multiple application of rule (R2)

Pitfall: Vertices appearing more than once in a configuration.

\section{How to find the rules?}\label{sec:rules}
\begin{itemize}
\item Comments on the choice of the initial charge, iteration between improving discharging rules and finding reducible configurations.
\item Rule values via linear programming, obstructions from dual solution. (What does this mean?)
\end{itemize}

\section{Easy Examples?}

\begin{itemize}
\item $3$-colorability with no 4 to 9 cycles
\item (say from Cranston and West), e.g. $5/2$ coloring for planar graphs?
\end{itemize}

\section{Trick: A Precolored Face}

Use precolored face to get rid of short cycles. Example: no $4$ to $7$ (or $8$?), $3$-coloring?
% averaging on long faces
 
\section{The Short Discharging Proof of Grotzsch's theorem (Zdenek's proof)}

\section{On the $4$-color-theorem?}

\chapter{List Coloring}

% LP

\section{Overview of the Method}
\begin{itemize}
\item definitions and why useful
\item Even cycles are $2$-choosable; antimatching (Erdos, Rubin, Taylor); Brooks' for list?
\end{itemize}

\section{Book Proof: $5$-choosability of planar graphs}

\begin{itemize}
\item (What is the second proof?!)
\item Comments on choice of the assumptions (if we can come up with something profound on the topic :-).
\end{itemize}

\section{The Short List-Coloring Proof of Grotzsch's theorem}

using $|P|\le 6$ and $|P|\le 4$ if allowing one adjacent pair of lists of size two

\chapter{Recoloring?}

% ZD

\section{Overview of the Method}

\begin{itemize}
\item Example: 3-colorability of quadrangulations of torus that contain several disjoint even non-contactible cycles
(cut, $2$-color the resulting bipartite graph, recolor near one of the faces created by cutting).
\end{itemize}

\section{Kempe Chains: proof of $5$-color theorem}

\section{Proof of Vizing's theorem}

\part{Coloring Graphs on Surfaces}

\chapter{Motivation and Problems}

% LP

\section{Euler's formula for Surfaces and the Heawood bound}

\section{The Modern View}

\section{Algorithms, Locally Planar Graphs and Critical Graphs}

\section{Precolored Vertices, Crossings, and Many Colorings}

\chapter{Critical Graphs}

% ZD

\section{Definitions, Motivations and Examples}

\begin{itemize}
\item Definitions.
\item Motivation for bounding the size and the structure of critical graphs.
\begin{itemize}
\item Algorithmic remarks---complexity of finding a fixed subgraph; does not directly
give a coloring if one exists.
\end{itemize}
\item Examples: coloring embedded graphs with at least $6$ colors.
\end{itemize}

\section{A History of Results}

Survey of the bounds for size and desity, known full lists of critical graphs.

Key Idea: Criticality with respect to a subgraph.

\section{Bounding the Size of a Critical Graph}

Overview Methods to bound size of a critical graph:
\begin{itemize}
\item Reductions increasing a properly defined weight (example: linear bound on the size of graphs in disk
critical for $5$-coloring?).
\item Specific subgraphs at the precolored face coming from the nibbling method (example: quadratic (or linear?) bound
on the size of graphs in disk critical for $5$-list-coloring)?
\end{itemize}

\section{Increasing Weight via Discharging}

\section{Increasing Weight via Nibbling}

\chapter{Hyperbolicity}

% LP

\section{Weak hyperbolicity and its consequences.}

\subsection{Edgewidth of critical graphs. }
\subsection{Logarithmic radius and coloring algorithm.}
\subsection{Precoloring extension?}
\subsection{Number of colorings? }

\section{Proof of the Weak Hyperbolic Structure Theorem?}

\section{Strong hyperbolicity and the size of embedded critical graphs.}

\section{Locally Free}

Application to far apart (at least in plane). Prove the plane version via Steiner Trees.

\section{Overview of the known hyperbolic families and bounds.}


\chapter{Triangle-free Graphs and Quadrangulations}

% ZD

\section{Bounding the sum of lengths of $(\ge\!5)$-faces in $4$-critical triangle-free graphs (outline only).}
\section{Coloring quadrangulations and near-quadrangulations.}

\subsection{ Winding number.}
\begin{itemize}
\item The disk case, analysis of critical graphs via flows.
\end{itemize}

\subsection{The general case (statement + proof outline).}

\section{Applications of the theory:}
\subsection{Algorithmic (outline).}
\subsection{Graphs with high edgewidth.}
\subsection{Coloring with few uncolored vertices.}
\subsection{Distant anomalies, Havel's problem (outline).}

\part{Advanced Methods}

\chapter{Density of Critical Graphs and the Potential Method}

% LP

\section{Maximum average degree, degeneracy, embedded graphs.}
\section{Potential instead of mad.}
\section{Short Potential Proof of Brook's Theorem}
\section{The Short Potential Proof of Grotzsch's Theorem}
\section{Outline of general $k$?}
\section{Survey of other results.}

\chapter{Flows and Orientations}
\section{Definitions and Motivation}
\section{Flow-Coloring duality and Tutte's Flow Conjectures}
\section{The Short Flow Proof of Grotzsch's Theorem}
\section{Thomassen's proof of the weak $3$-flow conjecture}
\section{Thomassen-inspired proof of $6$-flow?}

\chapter{Probabilistic Methods}
\section{Probabilistic Tools: Local Lemma and Concentration Inequalities}
\section{Nibble Method: Short Proof of Kim's result}
\section{Outline of Johansson?}
\section{Reed's Conjecture}
\subsection{Density Lemma}
\subsection{Outline of Sparsity Lemma}

\part{Variants of Colorings}

\chapter{More on List Coloring?}

Maybe?

\section{Orientations? Paintability?}
\section{Ohba's?}

\chapter{Edge Coloring}

Maybe?

Do we want to have this chapter?  Or possibly we could refocus it on the
sparse setting (cubic graphs, planar graphs -- Seymour's conjecture)?

\section{Vizing Fans, Kierstead Paths}
\section{Tashkinov Trees and the Goldberg-Seymour Conjecture}
\section{List Coloring Conjecture for Bipartite Graphs (Galvin's Proof)}
\section{List Coloring Conjecture: Outline of Kahn's Proof?}

\chapter{Correspondence Coloring}

% LP

\section{Definitions and Motivation}

Definition and properties of correspondence coloring.

\section{Results on list coloring via correspondence coloring}

\begin{itemize}
\item More reducible configurations: $3$-list-coloring graphs without $4$-, \ldots, $8$-cycles.
\item Restoring uniformity in probabilistic arguments: Sparse graphs of bounded maximum degree.
\end{itemize}

\section{Old Results through the lens of Corr. Coloring}

Corr. coloring tied to average degree (Berhnetsyn's Proof)

\section{Open Questions}

Lift matching number? (i.e. special case of correspondence edge coloring)

\chapter{Circular Coloring and Acyclic Coloring?}

Maybe?

\part{Coloring and Structure}

\chapter{Excluding Minors}

Maybe?

Results on graphs avoiding a minor:


\section{Decomposition to two low tree-width graphs, $2$-approximation algorithm.}
\section{Hadwiger's conjecture and its relaxations}
(colors induce subgraphs of bounded maximum degree, maximum component size).
\section{Hajos' conjecture and counterexamples}

\chapter{Forbidding Subgraphs}

Maybe?

\section{Strong Perfect Graph Theorem}
\section{Gyarfas Conj + $\chi$-boundedness}
\section{Erdos-Hajnal}

\part{Open Problems}

\begin{itemize}
\item 4-colorability (surfaces, planar + edge, one crossing, \ldots).
\item Characterization of planar graphs with $\alpha(G)=n/4$, Eulerian triangulations with $\alpha(G)=n/3$.
\item $4$-colorability of Eulerian triangulations of surfaces.
\item $3$-colorability without $4$-, $5$-, $6$-cycles.
\item Improving the distance in Havel's conjecture, constants in the bounds on critical graphs.
\item Algorithmic: extending a precoloring of an arbitrarily long face in triangle-free planar graphs.
\item Vertex minors and $\chi$-boundedness.
\end{itemize}

\newpage
\pagestyle{plain}
\addcontentsline{toc}{chapter}{Index}
\printindex

\addcontentsline{toc}{chapter}{Bibliography}
\bibliographystyle{acm}
\bibliography{cb}

\end{document}
