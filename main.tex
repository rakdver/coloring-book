\documentclass[12pt,twoside,openright,a4paper]{book}
\usepackage{graphicx}
\usepackage{amsmath}
\usepackage{amsthm}
\usepackage{makeidx}
\makeindex

\newtheorem{theorem}{Theorem}[chapter]
\newtheorem{lemma}[theorem]{Lemma}
\newtheorem{corollary}[theorem]{Corollary}
\newtheorem{conjecture}{Conjecture}[chapter]

\begin{document}
\pagestyle{empty}
\begin{titlepage}

\begin{center}

\vspace{10cm}

{\large\bf A Coloring Book}

\end{center}
\end{titlepage}

\newpage

\pagestyle{plain}
\pagenumbering{roman}
\tableofcontents

\newpage

\pagestyle{headings}
\pagenumbering{arabic}
\setcounter{page}{1}

\chapter*{Introduction}
\addcontentsline{toc}{chapter}{Introduction}

\section{Scope of the book}

(sparse graphs, vertex coloring), target audience (PhD level,
familiarity with graph theory and basic coloring results expected, otherwise self-contained).

\section{Basic definitions}

 (surfaces + girth, maximum average degree,
minors). 

\section{Overview of Results}

Note: How much overview here?

% \input introduction.tex

\part{Standard Methods}

Overview of Standard Methods (with examples and comments on common tricks and pitfalls):

Grotzch's theorem as a non-trivial motivating example

\chapter{Reducible Configurations and Discharging}

\section{Overview of the Method}

\begin{itemize}
\item Discuss Euler's formula
\item Proof of $5$-color theorem
\item Example: $3$-colorability with no 4 to 9 cycles
\item Comments on the choice of the initial charge, iteration between improving discharging rules and finding reducible configurations.
\end{itemize}

\section{Easy Examples?}

\begin{itemize}
\item (say from Cranston and West), e.g. $5/2$ coloring for planar graphs?
\item Rule values via linear programming, obstructions from dual solution. (What does this mean?)
\end{itemize}

\section{Trick: A Precolored Face}

Use precolored face to get rid of short cycles. Example: no $4$ to $7$ (or $8$?), $3$-coloring?

Pitfall: Vertices appearing more than once in a configuration.
 
\section{The Short Discharging Proof of Grotzsch's theorem (Zdenek's proof)}

\section{On the $4$-color-theorem?}

\chapter{List Coloring}

\section{Overview of the Method}
\begin{itemize}
\item definitions and why useful
\item Even cycles are $2$-choosable; antimatching (Erdos, Rubin, Taylor); Brooks' for list?
\end{itemize}

\section{Book Proof: $5$-choosability of planar graphs}

\begin{itemize}
\item (What is the second proof?!)
\item Comments on choice of the assumptions (if we can come up with something profound on the topic :-).
\end{itemize}

\section{The Short List-Coloring Proof of Grotzsch's theorem}

using $|P|\le 6$ and $|P|\le 4$ if allowing one adjacent pair of lists of size two

\chapter{Recoloring?}

\section{Overview of the Method}

\begin{itemize}
\item Example: 3-colorability of quadrangulations of torus that contain several disjoint even non-contactible cycles
(cut, $2$-color the resulting bipartite graph, recolor near one of the faces created by cutting).
\end{itemize}

\section{Kempe Chains: proof of $5$-color theorem}

\section{Proof of Vizing's theorem}


\part{Coloring Graphs on Surfaces}

\chapter{Motivation and Problems}

\section{Euler's formula for Surfaces and the Heawood bound}

\section{The Modern View}

\section{Algorithms, Locally Planar Graphs and Critical Graphs}

Notation: ``$6$-critical'' vs ``critical for $5$-coloring''.  We use the latter.

\section{Precolored Vertices, Crossings, and Many Colorings}

\chapter{Critical Graphs}

\section{Definitions, Motivations and Examples}

\begin{itemize}
\item Definitions.
\item Motivation for bounding the size and the structure of critical graphs.
\begin{itemize}
\item Algorithmic remarks---complexity of finding a fixed subgraph; does not directly
give a coloring if one exists.
\end{itemize}
\item Examples: coloring embedded graphs with at least $6$ colors.
\end{itemize}

\section{A History of Results}

Survey of the bounds for size and desity, known full lists of critical graphs.

Key Idea: Criticality with respect to a subgraph.

\section{Bounding the Size of a Critical Graph}

Overview Methods to bound size of a critical graph:
\begin{itemize}
\item Reductions increasing a properly defined weight (example: linear bound on the size of graphs in disk
critical for $5$-coloring?).
\item Specific subgraphs at the precolored face coming from the nibbling method (example: quadratic (or linear?) bound
on the size of graphs in disk critical for $5$-list-coloring)?
\end{itemize}

\section{Increasing Weight via Discharging}

\section{Increasing Weight via Nibbling}
% \input critical.tex

\chapter{Hyperbolicity}
% \input hyperbolicity.tex

\section{Weak hyperbolicity and its consequences.}

\subsection{Edgewidth of critical graphs. }
\subsection{Logarithmic radius and coloring algorithm.}
\subsection{Precoloring extension?}
\subsection{Number of colorings? }

\section{Proof of the Weak Hyperbolic Structure Theorem?}

\section{Strong hyperbolicity and the size of embedded critical graphs.}

\section{Locally Free}

Application to far apart (at least in plane). Prove the plane version via Steiner Trees.

\section{Overview of the known hyperbolic families and bounds.}


\chapter{Triangle-free Graphs and Quadrangulations}
% \input quadrangulations.tex

\section{Bounding the sum of lengths of $(\ge\!5)$-faces in $4$-critical triangle-free graphs (outline only).}
\section{Coloring quadrangulations and near-quadrangulations.}

\subsection{ Winding number.}
\begin{itemize}
\item The disk case, analysis of critical graphs via flows.
\end{itemize}

\subsection{The general case (statement + proof outline).}

\section{Applications of the theory:}
\subsection{Algorithmic (outline).}
\subsection{Graphs with high edgewidth.}
\subsection{Coloring with few uncolored vertices.}
\subsection{Distant anomalies, Havel's problem (outline).}

\part{Advanced Methods}

\chapter{Density of Critical Graphs and the Potential Method}
% \input potential.tex

\section{Maximum average degree, degeneracy, embedded graphs.}
\section{Potential instead of mad.}
\section{Short Potential Proof of Brook's Theorem}
\section{The Short Potential Proof of Grotzsch's Theorem}
\section{Outline of general $k$?}
\section{Survey of other results.}

\chapter{Flows and Orientations}
\section{Definitions and Motivation}
\section{Flow-Coloring duality and Tutte's Flow Conjectures}
\section{The Short Flow Proof of Grotzsch's Theorem}
\section{Thomassen's proof of the weak $3$-flow conjecture}
\section{Thomassen-inspired proof of $6$-flow?}

\chapter{Probabilistic Methods}
\section{Probabilistic Tools: Local Lemma and Concentration Inequalities}
\section{Nibble Method: Short Proof of Kim's result}
\section{Outline of Johansson?}
\section{Reed's Conjecture}
\subsection{Density Lemma}
\subsection{Outline of Sparsity Lemma}

\part{Variants of Colorings}

\chapter{More on List Coloring?}
\section{Orientations? Paintability?}
\section{Ohba's?}

\chapter{Edge Coloring}

Do we want to have this chapter?  Or possibly we could refocus it on the
sparse setting (cubic graphs, planar graphs -- Seymour's conjecture)?

\section{Vizing Fans, Kierstead Paths}
\section{Tashkinov Trees and the Goldberg-Seymour Conjecture}
\section{List Coloring Conjecture for Bipartite Graphs (Galvin's Proof)}
\section{List Coloring Conjecture: Outline of Kahn's Proof?}

\chapter{Correspondence Coloring}
% \input listandcorr.tex

\section{Definitions and Motivation}

Definition and properties of correspondence coloring.

\section{Results on list coloring via correspondence coloring}

\begin{itemize}
\item More reducible configurations: $3$-list-coloring graphs without $4$-, \ldots, $8$-cycles.
\item Restoring uniformity in probabilistic arguments: Sparse graphs of bounded maximum degree.
\end{itemize}

\section{Old Results through the lens of Corr. Coloring}

Corr. coloring tied to average degree (Berhnetsyn's Proof)

\section{Open Questions}

Lift matching number? (i.e. special case of correspondence edge coloring)

\chapter{Circular Coloring and Acyclic Coloring?}


\part{Coloring and Structure}

\chapter{Excluding Minors}
% \input structure.tex

Results on graphs avoiding a minor:


\section{Decomposition to two low tree-width graphs, $2$-approximation algorithm.}
\section{Hadwiger's conjecture and its relaxations}
(colors induce subgraphs of bounded maximum degree, maximum component size).
\section{Hajos' conjecture and counterexamples}

\chapter{Forbidding Subgraphs}

\section{Strong Perfect Graph Theorem}
\section{Gyarfas Conj + $\chi$-boundedness}
\section{Erdos-Hajnal}

\part{Open Problems}
% \input open.tex

\begin{itemize}
\item 4-colorability (surfaces, planar + edge, one crossing, \ldots).
\item Characterization of planar graphs with $\alpha(G)=n/4$, Eulerian triangulations with $\alpha(G)=n/3$.
\item $4$-colorability of Eulerian triangulations of surfaces.
\item $3$-colorability without $4$-, $5$-, $6$-cycles.
\item Improving the distance in Havel's conjecture, constants in the bounds on critical graphs.
\item Algorithmic: extending a precoloring of an arbitrarily long face in triangle-free planar graphs.
\item Vertex minors and $\chi$-boundedness.
\end{itemize}

\newpage
\pagestyle{plain}
\addcontentsline{toc}{chapter}{Index}
\printindex

\addcontentsline{toc}{chapter}{Bibliography}
\bibliographystyle{acm}
\bibliography{cb}

\end{document}
