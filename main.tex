\documentclass[12pt,twoside,openright,a4paper]{book}
\usepackage{graphicx}
\usepackage{amsmath}
\usepackage{amsthm}
\usepackage{makeidx}
\makeindex

\newtheorem{theorem}{Theorem}[chapter]
\newtheorem{lemma}[theorem]{Lemma}
\newtheorem{corollary}[theorem]{Corollary}
\newtheorem{conjecture}{Conjecture}[chapter]

\begin{document}
\pagestyle{empty}
\begin{titlepage}

\begin{center}

\vspace{10cm}

{\large\bf A Coloring Book}

\end{center}
\end{titlepage}

\newpage

\pagestyle{plain}
\pagenumbering{roman}
\tableofcontents

\newpage

\pagestyle{headings}
\pagenumbering{arabic}
\setcounter{page}{1}

\chapter*{Introduction}
\addcontentsline{toc}{chapter}{Introduction}
\begin{itemize}
\item Scope of the book (sparse graphs, vertex coloring), target audience (PhD level,
familiarity with graph theory and basic coloring results expected, otherwise self-contained).
\item Basic definitions and overview of results (surfaces + girth, maximum average degree,
minors).
\item Notation question: ``$6$-critical'' vs ``critical for $5$-coloring''?
\end{itemize}
% \input introduction.tex

\chapter{Methods}
\begin{itemize}
\item Standard methods (with examples and comments on common tricks and pitfalls):
\begin{itemize}
\item Reducible configurations and discharging.
\begin{itemize}
\item Example: ???
\item Comments on the choice of the initial charge, iteration between improving discharging rules and finding reducible configurations.
\item Rule values via linear programming, obstructions from dual solution.
\item Trick with precolored face to get rid of short cycles.
\item Example: 3-colorability of planar graphs without 4-, \ldots, 8-cycles?
\end{itemize}
\item Nibbling method.
\begin{itemize}
\item Examples: Two proofs of 5-choosability of planar graphs.
\item Comments on choice of the assumptions (if we can come up with something profound on the topic :-).
\end{itemize}
\item Recoloring.
\begin{itemize}
\item Example: 3-colorability of quadrangulations of torus that contain several disjoint even non-contactible cycles
(cut, $2$-color the resulting bipartite graph, recolor near one of the faces created by cutting).
\end{itemize}
\end{itemize}
\end{itemize}
% \input methods.tex

\chapter{Critical Graphs}

\begin{itemize}
\item Definitions.
\item Motivation for bounding the size and the structure of critical graphs.
\begin{itemize}
\item Algorithmic remarks---complexity of finding a fixed subgraph; does not directly
give a coloring if one exists.
\end{itemize}
\item Examples: coloring embedded graphs with at least $6$ colors.
\item Survey of the bounds for size and desity, known full lists of critical graphs.
\item Criticality with respect to a subgraph.
\item Methods to bound size of a critical graph.
\begin{itemize}
\item Reductions increasing a properly defined weight (example: linear bound on the size of graphs in disk
critical for $5$-coloring?).
\item Specific subgraphs at the precolored face coming from the nibble method (example: quadratic (or linear?) bound
on the size of graphs in disk critical for $5$-list-coloring)?
\end{itemize}
\end{itemize}
% \input critical.tex

\chapter{Hyperbolicity}
% \input hyperbolicity.tex

\begin{itemize}
\item Weak hyperbolicity and its consequences.
\begin{itemize}
\item Edgewidth of critical graphs.
\item Logarithmic radius and coloring algorithm.
\item Precoloring extension.
\item Number of colorings.
\end{itemize}
\item Strong hyperbolicity and the size of embedded critical graphs.
\item Overview of the known hyperbolic families and bounds.
\end{itemize}

\chapter{Triangle-free Graphs and Quadrangulations}
% \input quadrangulations.tex

\begin{itemize}
\item Bounding the sum of lengths of $(\ge\!5)$-faces in $4$-critical triangle-free graphs (outline only).
\item Coloring quadrangulations and near-quadrangulations.
\begin{itemize}
\item Winding number.
\begin{itemize}
\item The disk case, analysis of critical graphs via flows.
\end{itemize}
\item The general case (statement + proof outline).
\end{itemize}
\item Applications of the theory:
\begin{itemize}
\item Algorithmic (outline).
\item Graphs with high edgewidth.
\item Coloring with few uncolored vertices.
\item Distant anomalies, Havel's problem (outline).
\end{itemize}
\end{itemize}

\chapter{Density of Critical Graphs and the Potential Method}
% \input potential.tex

\begin{itemize}
\item Maximum average degree, degeneracy, embedded graphs.
\item Potential instead of mad.
\item Example: density of 4-critical graphs.
\item Survey of other results.
\end{itemize}

\chapter{List and Correspondence Coloring}
% \input listandcorr.tex

\begin{itemize}
\item Definition and properties of correspondence coloring.
\item Results on list coloring via correspondence coloring:
\begin{itemize}
\item More reducible configurations: $3$-list-coloring graphs without $4$-, \ldots, $8$-cycles.
\item Restoring uniformity in probabilistic arguments: Sparse graphs of bounded maximum degree.
\end{itemize}
\end{itemize}

\chapter{Coloring and Structure}
% \input structure.tex

\begin{itemize}
\item Results on graphs avoiding a minor:
\begin{itemize}
\item Decomposition to two low tree-width graphs, $2$-approximation algorithm.
\item Hadwiger's conjecture and its relaxations (colors induce subgraphs of bounded maximum degree, maximum component size).
\end{itemize}
\item Hajos' conjecture and counterexamples.
\item Forbidden induced subgraphs, bounds + algorithms for chromatic number, $\chi$-boundedness?
\end{itemize}

\chapter{Open Problems}
% \input open.tex

\begin{itemize}
\item 4-colorability (surfaces, planar + edge, one crossing, \ldots).
\item Characterization of planar graphs with $\alpha(G)=n/4$, Eulerian triangulations with $\alpha(G)=n/3$.
\item $4$-colorability of Eulerian triangulations of surfaces.
\item $3$-colorability without $4$-, $5$-, $6$-cycles.
\item Improving the distance in Havel's conjecture, constants in the bounds on critical graphs.
\item Algorithmic: extending a precoloring of an arbitrarily long face in triangle-free planar graphs.
\item Vertex minors and $\chi$-boundedness.
\end{itemize}

\newpage
\pagestyle{plain}
\addcontentsline{toc}{chapter}{Index}
\printindex

\addcontentsline{toc}{chapter}{Bibliography}
\bibliographystyle{acm}
\bibliography{cb}

\end{document}
